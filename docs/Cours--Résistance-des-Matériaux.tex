% Options for packages loaded elsewhere
\PassOptionsToPackage{unicode}{hyperref}
\PassOptionsToPackage{hyphens}{url}
%
\documentclass[
]{book}
\usepackage{lmodern}
\usepackage{amsmath}
\usepackage{ifxetex,ifluatex}
\ifnum 0\ifxetex 1\fi\ifluatex 1\fi=0 % if pdftex
  \usepackage[T1]{fontenc}
  \usepackage[utf8]{inputenc}
  \usepackage{textcomp} % provide euro and other symbols
  \usepackage{amssymb}
\else % if luatex or xetex
  \usepackage{unicode-math}
  \defaultfontfeatures{Scale=MatchLowercase}
  \defaultfontfeatures[\rmfamily]{Ligatures=TeX,Scale=1}
\fi
% Use upquote if available, for straight quotes in verbatim environments
\IfFileExists{upquote.sty}{\usepackage{upquote}}{}
\IfFileExists{microtype.sty}{% use microtype if available
  \usepackage[]{microtype}
  \UseMicrotypeSet[protrusion]{basicmath} % disable protrusion for tt fonts
}{}
\makeatletter
\@ifundefined{KOMAClassName}{% if non-KOMA class
  \IfFileExists{parskip.sty}{%
    \usepackage{parskip}
  }{% else
    \setlength{\parindent}{0pt}
    \setlength{\parskip}{6pt plus 2pt minus 1pt}}
}{% if KOMA class
  \KOMAoptions{parskip=half}}
\makeatother
\usepackage{xcolor}
\IfFileExists{xurl.sty}{\usepackage{xurl}}{} % add URL line breaks if available
\IfFileExists{bookmark.sty}{\usepackage{bookmark}}{\usepackage{hyperref}}
\hypersetup{
  pdftitle={Travaux dirigés: Résistances de Matériaux -- ENSGSI},
  pdfauthor={Olivier FARGE \& Fabio CRUZ},
  hidelinks,
  pdfcreator={LaTeX via pandoc}}
\urlstyle{same} % disable monospaced font for URLs
\usepackage{longtable,booktabs}
\usepackage{calc} % for calculating minipage widths
% Correct order of tables after \paragraph or \subparagraph
\usepackage{etoolbox}
\makeatletter
\patchcmd\longtable{\par}{\if@noskipsec\mbox{}\fi\par}{}{}
\makeatother
% Allow footnotes in longtable head/foot
\IfFileExists{footnotehyper.sty}{\usepackage{footnotehyper}}{\usepackage{footnote}}
\makesavenoteenv{longtable}
\usepackage{graphicx}
\makeatletter
\def\maxwidth{\ifdim\Gin@nat@width>\linewidth\linewidth\else\Gin@nat@width\fi}
\def\maxheight{\ifdim\Gin@nat@height>\textheight\textheight\else\Gin@nat@height\fi}
\makeatother
% Scale images if necessary, so that they will not overflow the page
% margins by default, and it is still possible to overwrite the defaults
% using explicit options in \includegraphics[width, height, ...]{}
\setkeys{Gin}{width=\maxwidth,height=\maxheight,keepaspectratio}
% Set default figure placement to htbp
\makeatletter
\def\fps@figure{htbp}
\makeatother
\setlength{\emergencystretch}{3em} % prevent overfull lines
\providecommand{\tightlist}{%
  \setlength{\itemsep}{0pt}\setlength{\parskip}{0pt}}
\setcounter{secnumdepth}{5}
\usepackage{booktabs}
\usepackage{amsmath}
\ifluatex
  \usepackage{selnolig}  % disable illegal ligatures
\fi
\usepackage[]{natbib}
\bibliographystyle{apalike}

\title{Travaux dirigés: Résistances de Matériaux -- ENSGSI}
\author{Olivier FARGE \& Fabio CRUZ}
\date{2021-01-10}

\begin{document}
\maketitle

{
\setcounter{tocdepth}{1}
\tableofcontents
}
\hypertarget{introduction}{%
\chapter{Introduction}\label{introduction}}

\hypertarget{intro}{%
\section{Plan du cours}\label{intro}}

\begin{enumerate}
\def\labelenumi{\arabic{enumi}.}
\tightlist
\item
  Introduction
\end{enumerate}

\begin{itemize}
\tightlist
\item
  Modalités de déroulemen et validation du module RDM
\item
  La Mécanique, la Résistances des Matériaux
\item
  Dimensionnement des structures
\end{itemize}

\begin{enumerate}
\def\labelenumi{\arabic{enumi}.}
\tightlist
\item
  Notions sur les torseurs
\item
  Géometrie des poutres
\item
  Statique
\item
  Expérience fondamentale
\item
  Bilan des hypothèses
\item
  Applications: Sollicitations simples
\end{enumerate}

\begin{itemize}
\tightlist
\item
  Traction - compression
\item
  Flexion
\end{itemize}

\begin{enumerate}
\def\labelenumi{\arabic{enumi}.}
\tightlist
\item
  Dimensionnement
\end{enumerate}

\hypertarget{equipe-puxe9dagogique}{%
\section{Equipe Pédagogique}\label{equipe-puxe9dagogique}}

\hypertarget{suxe9quences-denseignement-du-module}{%
\section{Séquences d'enseignement du module}\label{suxe9quences-denseignement-du-module}}

\begin{itemize}
\tightlist
\item
  12 séances de cours
\item
  11 séances de Travaux dirigés
\item
  1 Conférence industrielle
\end{itemize}

\hypertarget{ruxe9partition-des-suxe9quences-denseignement}{%
\section{Répartition des séquences d'enseignement}\label{ruxe9partition-des-suxe9quences-denseignement}}

\hypertarget{td-1-exercices-dapplication-sur-la-notions-de-torseur}{%
\chapter{TD 1: Exercices d'Application sur la notions de torseur}\label{td-1-exercices-dapplication-sur-la-notions-de-torseur}}

\hypertarget{exercise-1}{%
\section{Exercise 1}\label{exercise-1}}

Soit \((O;i, j, k)\) un repère orthonormé direct. On note \((x, y, z)\) les coordonnés du point \(P\) et on considère le champ de vecteurs \(H(P)\) suivant:

\[
H(P) = 
\begin{bmatrix}
-w[(y - y_{0}) \cos(\theta) + z\sin(\theta) ] \\
-w(x - x_{0}) \cos(\theta) \\
-w(x - x_{0}) \sin(\theta) + \frac{v}{\cos(\theta)} \\
\end{bmatrix}
\]
où \(x\)

\hypertarget{methods}{%
\chapter{Methods}\label{methods}}

We describe our methods in this chapter.

\hypertarget{applications}{%
\chapter{Applications}\label{applications}}

Some \emph{significant} applications are demonstrated in this chapter.

\hypertarget{example-one}{%
\section{Example one}\label{example-one}}

\hypertarget{example-two}{%
\section{Example two}\label{example-two}}

\hypertarget{final-words}{%
\chapter{Final Words}\label{final-words}}

We have finished a nice book.

  \bibliography{book.bib,packages.bib}

\end{document}
